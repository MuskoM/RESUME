\documentclass[10pt, a4paper]{article} % The document class with options
\usepackage[default]{lato} % Lato font
\usepackage{fontspec} % Use fontspec for lualatex or xelatex
% \usepackage[polish]{babel} % Localization for polish
\usepackage[margin=0.75in]{geometry} % Set margins to 1 inch
\usepackage{hyperref}
\usepackage[parfill]{parskip} % Empty line between paragraph instead of indent
\usepackage{fancyhdr}

%==============================================================================
% FOOTER CONFIGURATION
%==============================================================================
\pagestyle{fancy} % Activates the fancy page style
\fancyhf{} % Clears all default header and footer fields
\renewcommand{\headrulewidth}{0pt} % Removes the line under the header
\renewcommand{\footrulewidth}{0.4pt} % Adds a thin line above the footer

% Defines the centered footer content for every page
\fancyfoot[C]{\footnotesize\itshape I hereby consent to the processing of my personal data included in my CV for the purposes of the recruitment process, in accordance with the General Data Protection Regulation (EU) 2016/679.}

\hypersetup{
    colorlinks=true,
    linkcolor=blue,
    urlcolor=blue,
}

\begin{document}
\begin{center}
    {\Huge \textbf{Mateusz Muśko}} \\
    \vspace{0.1cm}
    \href{mailto:job@musko.codes}{job@musko.codes} $|$
    \href{https://musko.codes}{musko.codes} $|$
    \href{https://github.com/MuskoM}{github.com/MuskoM} \\
\end{center}
\vspace{0.5cm}

%==============================================================================
% SUMMARY
%==============================================================================
\section*{Professional Summary}
\hrule
\vspace{0.2cm}
Results-driven Software Developer with over 3 years of experience in architecting and delivering full-stack solutions for complex R\&D projects at Fujitsu. Proven expertise in backend development with Python and FastAPI, coupled with extensive experience in cloud-native technologies across AWS and Azure. A certified cloud practitioner skilled in building CI/CD pipelines, automating infrastructure with IaC, and leading development teams.
\vspace{0.3cm}

\textbf{Core Technologies:} Python | FastAPI | RESTful APIs | PostgreSQL | MongoDB | Docker | AWS | Azure

\vspace{0.5cm}

%==============================================================================
% WORK EXPERIENCE
%==============================================================================
\section*{Work Experience}
\hrule
\subsection*{Software Developer}
\textsc{Fujitsu} \textit{February 2023 – Present}
\begin{itemize}
    \itemsep -0.5em % Optional: Adjusts spacing between items
    \item Delivered features across four major R\&D projects: Guided Workflow Creation, Flight Plan Optimization, an IoT Device Management System, and a Video Behaviour Analytics platform.
    \item For the Video Analytics project, developed custom object detection models (YOLOx) and behavior detection models using a Few-Shot Learning approach for CCTV analysis.
    \item Led a team of 3 developers on the Guided Workflow Creator project, establishing the initial configuration and developing the core feature to dynamically modify workflow diagrams based on user questionnaire responses.
    \item Engineered and documented the deployment of an Amazon EKS cluster to host the Eclipse Hono IoT framework and designed a data funneling service to stream IoT data into a Kafka pipeline.
    \item Designed system architecture and infrastructure diagrams for multiple projects; proposed and prototyped key features including a WebSocket-based alert system and an NLP-driven flow conversion tool.
    \item Engineered a full-stack feature for the Flight Plan Optimization tool to merge booking lists, which included the backend API endpoint and a new frontend interface with over 90\% test coverage.
    \item Built a suite of MLOps tools to support the machine learning lifecycle, including a model performance visualization tool, data aggregation scripts, and video preprocessing utilities.
\end{itemize}

\subsection*{Associate Python Developer}
\textsc{Fujitsu} \textit{January 2022 – January 2023}
\begin{itemize}
    \itemsep -0.5em % Optional: Adjusts spacing between items
    \item Delivered features across two key R\&D projects: an AI-driven healthcare application and a cloud-native security platform for automotive fleets that displayed real-time vehicle alerts on an interactive map.
    \item Authored Infrastructure as Code (IaC) using AWS CloudFormation to automate the deployment of a massive test data generation module and an ElastiCache (Redis) instance for caching.
    \item Engineered horizontal scaling policies for AWS Fargate and Redis to ensure the high availability and performance of data services under heavy load.
    \item Remediated security vulnerabilities by correcting improper AWS IAM rule assignments and established a CI pipeline to automate testing for an external-facing API.
    \item For the healthcare platform, designed and developed multiple Python backend services to manage hospital resources, employee schedules, and predictive patient care plans.
    \item For the fleet management platform, designed and built frontend views to display vehicle alerts with severity levels and analytics charts, and integrated new components with the primary map display.
    \item Proposed and executed a significant UI redesign for several dashboard components, including status, analytics, and detail views, to improve user experience and clarity.
\end{itemize}

\subsection*{Assistant Software Developer}
\textsc{Fujitsu}, \textit{September 2021 – December 2021}
\begin{itemize}
    \itemsep -0.3em % Optional: Reduces space between list items for a tighter look
    \item Contributed to a strategic R\&D project to integrate Fujitsu's proprietary Digital Annealing (quantum-inspired computing) technology with the Microsoft Azure platform, making it accessible as a cloud service.
    \item Developed a key module for user billing and resource metering using Python, Celery, and Redis, ensuring accurate tracking of service consumption.
    \item Analyzed, debugged, and resolved issues within a large-scale, asynchronous Python codebase (using FastAPI and asyncio), improving overall application stability.
    \item Gained practical experience with cloud-native database solutions, including Azure CosmosDB and PostgreSQL, for managing application and user data.
    \item Operated within an Agile framework using Azure DevOps for sprint planning, task management, and version control (Git).
\end{itemize}

%==============================================================================
% TECHNICAL SKILLS
%==============================================================================
\section*{Technical Skills}
\hrule
\vspace{0.2cm}
\begin{tabular}{@{}p{0.25\textwidth} p{0.75\textwidth}}
    \textbf{Languages} & Python, TypeScript, C\#, SQL \\
    \textbf{Backend} & FastAPI, Flask, Pydantic, Celery \\
    \textbf{AI \& ML} & Pandas, Numpy, Scikit-learn, OpenAI (GPT-4), YOLOx, Few-Shot Learning \\
    \textbf{Frontend} & Vue.js, TypeScript, Vitest \\
    \textbf{Cloud \& DevOps} & AWS (Fargate, SQS, EC2, S3, CDK), Azure (Functions, DevOps), Docker, EKS, Kafka, RabbitMQ, CI/CD \\
    \textbf{Databases} & PostgreSQL, DynamoDB, Redis, MongoDB \\
    \textbf{Tools} & Git, Jira, Linux (openSUSE, Ubuntu, Debian), vim, neovim \\
\end{tabular}

\vspace{0.5cm}

%==============================================================================
% EDUCATION
%==============================================================================
\section*{Education}
\hrule
\vspace{0.2cm}
\textbf{Master of Science - Intelligent Systems}, Politechnika Białostocka \hfill \textit{October 2023} \\
\textbf{Bachelor of Science - Software Engineering}, Politechnika Białostocka \hfill \textit{March 2022} \\

\vspace{0.5cm}

%==============================================================================
% CERTIFICATIONS
%==============================================================================
\section*{Certifications}
\hrule
\vspace{0.2cm}
\textbf{AI\_devs 2 Reloaded}, Brave Courses \hfill \textit{April 2024} \\
\textbf{Certified Cloud Practitioner}, AWS \hfill \textit{October 2022} \\
\textbf{Azure Fundamentals}, Microsoft \hfill \textit{April 2022} \\

\vspace{0.5cm}

%==============================================================================
% LANGUAGES
%==============================================================================
\section*{Languages}
\hrule
\vspace{0.2cm}
\begin{tabular}{@{}ll}
    \textbf{Polish} & Native \\
    \textbf{English} & Full Professional Proficiency \\
\end{tabular}

\end{document}

